\newcommand{\HRule}{\rule{\linewidth}{0.5mm}} % Definition epaisseur des lignes horizontales

\vspace*{1cm}

\begin{center} % title
	\HRule \\[0.2cm] %vertical line
	\Large
	\textbf{SETR-ME2 Langages matériels, réutilisation et intégration}\\ %Titre
	\vspace{1cm}
Compte rendu mini-projet \\

	\large
	\HRule \\[1.5cm] %vertical line
	\normalsize
	Louison Gouy et Téo Biton\\
	\today %date
\end{center}

\begin{figure}[H] %logos
	\centering
	\includegraphics[width=0.8\linewidth]{PolytechNantes}
\end{figure}

\vspace{2cm}

\begin{center}\large %School
	\textsc{École Polytech de Nantes}\\
	\textsc{Electronique et Technologies Numériques}
\end{center}

\vspace{2cm}

\noindent
Enseignants référents : Sébastien PILLEMENT et Maria MENDEZ REAL
\vspace{2cm}

\begin{abstract}
En se basant sur une IP de cellule RAM en VHDL, il s'agit de construire un bloc mémoire générique avec un décodeur pour accéder à un nombre de RAM à définir.
Ensuite, ce bloc mémoire doit être connecté à un modèle de processeur simple dans le but d'exécuter un programme assembleur. L'objectif final est d'avoir un top module
le plus générique possible et qu'il soit synthétisable.
\end{abstract}

\thispagestyle{empty}
