\section{Conclusion}

\gap

\indent Pour conclure, ce mini-projet nous a permis de fortement consolider nos compétences en langages de description matérielle, principalement par la découverte du langage Verilog.
De plus, c'était un premier pas vers des projets d'intégration à grande échelle où la réutilisation et la généricité des composants est de mise.
L'utilisation d'outils open source nous a poussé à être curieux et surtout confronté à des problèmes classiques lors la mise en place d'environnement de développement partagés: gérer correctement les versions des logiciels utilisés et passer par un outil de gestion de versions comme Git. \\
\indent Valider le fonctionnement global du sytème a été d'une grande satisfaction tout comme parvenir à synthétiser le projet.
Le temps investi à formaliser les bancs de test a permis de gagner un temps considérable, notamment lors des phases d'intégration qui n'ont généré que peu de bugs.
La prochaine étape aurait été de configurer un FPGA avec un modèle synthétisé et d'y charger des instructions.