\section{Introduction}

\gap

\indent L'objectif de ce mini-projet est de concevoir un système à base de microprocesseur en réutilisant des blocs de composants de base. La première partie du TP consiste à reprendre le code d'une RAM simple décrite en VHDL en la rendant générique.
Ensuite, il faut créer un bloc contenant plusieurs instances de cette RAM et un décodeur pour y accéder.
Enfin, la dernière partie du projet consiste à intégrer ce bloc mémoire à un système comprenant un processeur décrit en Verilog. \\
\indent Ce TP a pour objectifs d'initier les étudiants aux notions de généricité et de réutilisation d'\gls{IP}, tout en leur permettant de renouer avec le VHDL et de découvrir le Verilog.
De plus, ils seront amenés à intégrer les différents composants entre eux de sorte à produire un module global utilisable. \\
\indent Pour mener à bien ce TP, nous avons fait le choix d'une approche atypique : pour limiter les contraintes liées à l'utilisation de \textit{ModelSim} sur les machines de l'école, nous avons travaillé avec des outils de compilation et de simulation Open Source disponibles sur nos machines.
Ainsi, nous avons utilisé \textit{ghdl 2.0.0} pour la compilation et la simulation des codes VHDL et \textit{gtkwave 3.3.104} pour l'affichage.
De plus, pour élargir notre champs de compétences au delà des attendus du TP, nous avons défini dès le départ des modules ou langages à utiliser pour les bancs de tests.
Pour compiler et simuler les codes VHDL, nous avons utilisé \textit{Cocotb}, librairie Python permettant de tester et simuler des codes en langages de description matérielle.
Pour simuler le code en \textit{Verilog} du processeur, nous avons défini des bancs de test en C++.
Enfin, pour simuler le top module intégrant le bloc processeur et le bloc mémoire, nous avons utilisé \textit{ModelSim}, moins contraignant pour simuler un système à langages mixtes.
L'élaboration RTL à quant à elle été réalisée sous \textit{Vivado}.
Cette approche peut paraitre contreproductive dans un cadre industriel mais, dans ce contexte pédagogique, se familiariser avec différents outils prend plus de sens. \\
\indent Pour chaque étape, les choix technologiques ainsi que des preuves de fonctionnement en simulation seront montrées.
Les codes sources sont fournis dans une archive annexe et dans le projet GitHub https://github.com/loulou4418/Reuse\_indegration.

